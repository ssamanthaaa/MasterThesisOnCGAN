\newpage
\pagenumbering{Roman}
\selectlanguage{english}%
\begin{abstract}
Over the last decade, Generative Adversarial Networks (GANs) have made significant progress in image generation. Their capacity to generate high-quality, complex images by learning the distribution of the input data and generating new samples that are indistinguishable from real ones opened up new possibilities for research advancements and innovations. Among all the different opportunities created by GANs, realistic images of human faces, modification of image styles, voice and text generation stand out.

\noindent The research focuses on exploring the concept of conditional Generative Adversarial Networks, which are neural networks capable of being conditioned during the image generation process. This conditioning allows to have the control over the content of the generated image, and it is used to produce a realistic face image from an input sketch.

\noindent The thesis is guided by the research question: “\textit{Can people accurately distinguish between a conditionally generated photorealistic image and real images of similar appearance?}”.

\noindent To address the question, a GAN model has been trained on a vast dataset of face sketches and the corresponding images. Afterwards, the performance of the model has been qualitatively evaluated through human assessment. 

\noindent The models and techniques used in this thesis already exist in the literature, however this study differs in the way they have been trained. For the purpose of the thesis, the generative model has been trained on a novel dataset specifically created using computer vision techniques. 

\noindent Finally, the results and findings are presented, followed by the encountered limitations and the challenges faced by GANs in generating realistic images.
\end{abstract}


\newpage
\selectlanguage{italian}%
\begin{abstract}
Nell’ultimo decennio, le reti generative avversarie (GANs) hanno fatto progressi significativi nella generazione di immagini. La loro capacità di generare immagini complesse e di alta qualità apprendendo la distribuzione dei dati di input e generando nuovi campioni indistinguibili da quelli reali offre nuove possibilità di progresso e innovazione nella ricerca. Tra tutte le diverse opportunità create dalle GAN, spiccano la generazione di immagini realistiche di volti umani, la modifica dello stile di un’immagine, la generazione di voce e testo.

\noindent La ricerca si concentra sull’esplorazione del concetto di reti generative avversarie condizionali, ossia reti neurali in grado di essere condizionate durante il processo di generazione di un’immagine. Questo condizionamento permette di avere il controllo sul contenuto dell’immagine generata, ed è usato per produrre immagini di volti realistici partendo da un disegno.

\noindent La tesi è guidata dalla domanda di ricerca: “\textit{Possono le persone distinguere accuratamente tra un’immagine fotorealistica generata in modo condizionale e immagini reali di aspetto simile?}”.

\noindent Per rispondere a questa domanda è stato addestrato un modello GAN su un vasto dataset di disegni di volti e le immagini corrispondenti. Successivamente le prestazioni del modello sono state valutate qualitativamente attraverso la valutazione effettuata da diverse persone. 

\noindent I modelli e le tecniche utilizzati in questa tesi esistono già in letteratura, tuttavia lo studio si differenzia nel modo in cui sono stati addestrati. Ai fini della tesi, il modello generativo è stato allenato su un nuovo set di dati appositamente creato utilizzando tecniche di computer vision. 

\noindent Infine, vengono presentati i risultati e le scoperte, seguiti dalle limitazioni incontrate e dalle sfide affrontate dalle GAN nella generazione di immagini realistiche.

\end{abstract}

