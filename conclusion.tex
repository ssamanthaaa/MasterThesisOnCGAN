\newpage
\section{Conclusions}
\label{sec:conclusion}
In this thesis, a generative adversarial network (GAN) is trained on a large dataset of face images to explore its capability in synthesizing photorealistic images from face sketches. Then the performance of the model was qualitatively assessed to evaluate the results obtained.

\noindent The results and progress attained throughout the process draw attention on the outstanding accomplishments of the model and its successfulness in generating photorealistic images from face sketches. Furthermore, the model’s achievements highlight the promising potential it can have if applied to various fields, such as digital art, entertainment, and criminal investigations.

\noindent However, despite its remarkable performance, the model still has certain drawbacks that need to be addressed. Firstly, it may not be able to generate photorealistic images for all types of sketches, particularly those that are incomplete or poorly-drawn. Secondly, it does not generate images of all races equally, factor that can lead to a biased output. Moreover, it turned out to be challenging to generate images of children and younger people, as the model tended to generate an adult face most of the time. Lastly, the model is currently not able to capture some specific features present in the input sketches, such as piercings, tattoos, and freckles, therefore, when these are present they are not reported in the portrait.\\

\noindent In addition, according to the analysis conducted on the qualitative assessment’s result, the ultimate outcome of the model highlights that the generated images are perceived to be realistic with a high degree of resemblance between sketches and generated portrait.\\

\noindent Future developments in this field can potentially involve a more in-depth exploration of the use of Stable Diffusion when applied to the generation of photorealistic images based on face sketches. Given the speed at which this area of research is evolving, several further techniques and approaches have the potential to be leveraged in the future to improve the performance of the current model. For example, additional researches can focus on addressing the identified limitations, and enhancing the model's ability to capture such undervalued details. The proposed approach can also be adapted and applied to other domains beyond face image synthesis, such as generating photorealistic images of objects and scenes from sketches.
Future researches can and should also explore the use of other evaluation metrics, such as quantitative measures and other user studies.\\

\noindent Finally, this research provides valuable insights on the potential of GANs in image generation and unwraps new possibilities for future research in this field. The results obtained in this study can be used as a foundation for further investigation and development of GAN-based models for various applications.